\section{Introduction}

Brain-computer interfaces (BCI) establish a direct pathway between the human brain and a computer via signal processing and decoding techniques. 
One classic paradigm of EEG is motor imagery (MI), in which its physiological basis is based on body movements or imagined movements that can produce $\alpha$ (8-13 Hz) and $\beta$ (13-30 Hz) event-related synchronization (ERS) and event-related desynchronization (ERD) rhythms in the motor-sensory areas of the brain \cite{wolpaw2013brain}. 
Recently, the deep learning (DL) model has been used for motor imagery classification. 
For example, EEGNet \cite{lawhern2018eegnet} is a compact convolutional neural network designed for EEG-based brain-computer interfaces that effectively extracts spatial-temporal features from EEG signals.
In any case, the paucity of data is a prevalent issue in the field of EEG classification, as it hinders the development and performance of DL models.
Consequently, a common symptom is overfitting, which reduces the model's accuracy and robustness on test set \cite{bilbao2017overfitting}. 

Data augmentation (DA) has been widely used to improve the robustness and accuracy of DL by artificially increasing the number of training data.
Traditional EEG data augmentation methods include \texttt{Noise Addition} \cite{wang2018data,parvan2019transfer,li2019channel}, fourier transform surrogates \cite{schwabedal2018addressing}, \texttt{Frequency Shift}ing \cite{rommel2021cadda,rommel2022data} and \texttt{\texttt{SmoothTimeMask}} \cite{mohsenvand2020contrastive}.
\texttt{Noise Addition} \cite{li2019channel,parvan2019transfer} adds random white noise to all channels.
Fourier transform surrogates \cite{schwabedal2018addressing} randomizes the Fourier-transform (FT) phases of temporal-spatial data and generates surrogates that approximate examples from the data-generating distribution. \texttt{Frequency Shift}ing \cite{rommel2021cadda,rommel2022data} randomly shifts the frequency spectrum on all channels.
Last, \texttt{\texttt{SmoothTimeMask}} \cite{mohsenvand2020contrastive} randomly masks consecutive time steps of the EEG signal and replaces them with zeros, in which the motivation is to force the model to disregard minor irrelevant events.

Recently, diffusion model \cite{ho2020denoising} was proposed which generates synthetic data based on Langevin dynamics.
These models naturally admitted a progressive lossy decompression scheme.
Diffusion mode has been used as a DA method to generate synthetic training data for skin disease classification \cite{akrout2023diffusion}, prostate cancer detection \cite{hao2021comprehensive}, chest X-ray imaging \cite{motamed2021data}, etc.

In this work, we demonstrated the use of the diffusion model for data augmentation.
Particularly, we developed our diffusion model based on \texttt{WaveGrad} \cite{chen2020wavegrad} as a DA method for motor imagery classification.
\texttt{WaveGrad} has been initially for audio waveform generation.
The proposed approach involves utilizing score matching \cite{song2020sliced, song2020improved} and diffusion probabilistic \cite{sohl2015deep, ho2020denoising} models to estimate gradients of the data density within a conditional model. 
Because \texttt{WaveGrad} is shown to be successful in waveform generation, we apply the technique in the EEG signal with sequence length adjusted.
The process involves initializing the model with a Gaussian white noise signal and subsequently improving the signal quality through an iterative process that utilizes a gradient-based sampler that is conditioned on the mel-spectrogram.
We evaluated the effect of the proposed method by performing DA on BCI Competition IV 2a \cite{brunner2008bci} with various sizes of synthetic data with five standards EEG MI models (EEGNet \cite{lawhern2018eegnet}, ATCNet \cite{altaheri2022physics}, EEG-ITNet \cite{salami2022eeg}, Deep ConvNet \cite{schirrmeister2017deep} and ShallowFBCSPNet \cite{schirrmeister2017deep}). 
The proposed method improves the performance of classification models and outperforms other traditional EEG data augmentation methods. 

